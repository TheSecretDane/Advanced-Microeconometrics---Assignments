\section{Cobb-Douglas production} \label{sec:intro}

In this assignment we want to test whether the Cobb-Douglas production function (CD) exhibits constant returns to scale (CRS) at a firm level using panel data on French manufacturing firms using classical unobserved effect methods. The Cobb-Douglas production function is given by,
\begin{align} \label{eq:CD}
    F(K,L) = A K^{\beta_K} L^{\beta_L}
\end{align}
where $K$ is capital, $L$ is labor, $A$ is total factor productivity (TFP) and $\beta_K, \beta_L$ are the output elasticities of capital and labor, respectively.
It follows that for CD to have CRS, it must be homogeneous of degree one, i.e. $\beta_K + \beta_L = 1$, as seen from,
\begin{align*} 
    F(\lambda K, \lambda L) &= A (\lambda K)^{\beta_K} (\lambda L)^{\beta_L} = \lambda^{\beta_K + \beta_L} A K^{\beta_K} L^{\beta_L} = \lambda F(K,L) \iff \beta_K + \beta_L = 1.
\end{align*} 
which forms the linear hypothesis we want to test i.e. $H_0: \beta_K + \beta_L = 1$ vs. $H_A: \beta_K + \beta_L \neq 1$.
While $K$ and $L$ are certainly important production inputs, we cannot hope to observe \textbf{all} input factors that make up production such as managerial quality or structure i.e. $A$ is unobservable. To investigate the functional form of the production function, we estimate simple linear panel models, to remove the time-invarying unobserved heterogeneity and assess their merits based on the required assumptions for proper statistical modelling.

We find evidence that the strict exogeneity assumption needed for consistent estimation of the partial effects of capital and labor, while controlling for time-constant fixed effects, $c_i$, is violated. Due to this endogeneity issue, inference should be interpreted with (a lot) of caution, where we nonetheless find that the French manufacturing firms does not exhibit CRS, but rather decreasing returns to scale. 

