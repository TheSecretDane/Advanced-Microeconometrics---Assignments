\section{Empirical results}

The data consists of $N = 441$ French manufacturing firms over 12 years $(T=12)$ from 1968 to 1979, and is thus a panel data set with $NT = 5292$ observations. The dataset is balanced.  

We estimate the models using the methods outlined in Section \ref{sec:methodology}. Table \ref{tab:autocorr_tests} reports the results of the autocorrelation tests with both models exhibiting significant first-order autocorrelation additional to the mechanically induced autocorrelation implied by the transformation(s), and as such we use cluster robust standard errors  - to ensure consistent variance estimation. The estimates along with robust standard errors are reported in Table \ref{tab:estimates}. Coefficient significance and sign are consistent between methods that is we find a positive partial effect on log deflated sales from increasing any of the inputs conditional on the unobserved heterogeneity, as one would except. With labor increases resulting in a factor $\approx 4.5$ and $\approx 8.7$ relative sales increase, in percent, respectively. 

\begin{table}[H] 
\centering
\caption{Fixed Effects Regression Results}
\begin{tabular}{lcccc}
\toprule
 & $\beta$  & Se & t-values & p-value \\
\midrule
$\ell $  & 0.6942 & 0.0147 & 47.2447 & 0.0000 \\
$k$ & 0.1546 & 0.0130 & 11.9311 & 0.0000 \\
\midrule
$R^2$ & \multicolumn{4}{l}{0.477} \\
$\sigma^2$ & \multicolumn{4}{l}{0.018} \\
\bottomrule
\end{tabular}
\label{tab:results}
\end{table}


\begin{table}[H]
\centering
\caption{Serial correlation tests on residuals}
\label{tab:autocorr_tests}
\begin{tabular}{lcc}
\hline
 & $\text{FD}$ & $\text{FE}$ \\
\hline
Lag residual ($\hat{e}_{it-1}$ / $\hat{\ddot{u}}_{it-1}$)
  & $\underset{(0.0148)}{-0.1987^{***}}$
  & $\underset{(0.0248)}{0.5316^{***}}$ \\
$t$-stat & -13.4493 & 25.1137 \\
\hline
\end{tabular}

\begin{flushleft}\footnotesize
Notes: Each column reports a regression of residuals on their first lag:
$\hat{e}_{it}=\rho \hat{e}_{it-1}+v_{it}$ for first-differenced (FD) residuals and
$\hat{\ddot{u}}_{it}=\rho \hat{\ddot{u}}_{it-1}+v_{it}$ for fixed-effects (FE) residuals.
Standard errors in parentheses, robust in the FE model. ${}^{***}p<0.01$. Note that we are testing different hypothesis as per the methodology section, but both that both test statistics are asymptotically normally distributed.
\end{flushleft}
\end{table}


However, for the estimates to carry any meaning, at all, we must test whether assumptions \textbf{FE(D).1} hold. We test strict exogeneity by adding leads of the regressors as described in the latter part of Section \ref{sec:methodology}. In the FE specifications all subsets of $\bm{\ddot{x}_{it+1}}$ are significant, whereas none of the $\bm{x_{it}}$ subsets are significant in the FD specifications (note the distinction between transformed leaded variables and levels in the FE vs FD  as per \cite{wooldridgeEconometricAnalysisCross2010}). Consequently, we would reject strict exogeneity for the FE model and fail to reject it for the FD model. The hypothesis for the joint tests are, 
\begin{align*}
    \bm{R} \bm{\beta} = \bm{r} \Longrightarrow = \begin{bmatrix}
0 & 0 & 1 & 0 \\
0 & 0 & 0 & 1
\end{bmatrix} \begin{bmatrix}
\beta_L \\ \beta_K \\ \delta_{L} \\ \delta_{K}
\end{bmatrix} = \begin{bmatrix}
0 \\ 0
\end{bmatrix} \\
H_0: \delta_L = \delta_K = 0 \quad \text{vs} \quad H_A: \text{any } \delta_i \neq 0, i=L, K
\end{align*}
This suggests that the strict exogeneity assumption on the \textit{population} might not hold, such that neither $\hat{\beta}_{\text{FE}}$ and $\hat{\beta}_{\text{FD}}$ are consistent, even if one finite sample model suggests otherwise which simply could be due to the violation having a certain structure which might get washed out in the transformation(s). The full set of results is reported in Table~\ref{tab:strict_exog_test}. 

\begin{table}[H]
\centering
\caption{Strict exogeneity tests}
\label{tab:strict_exog_test}
\begin{tabular}{lcccccc}
\hline
 & $\text{FE}_1$ & $\text{FE}_2$ & $\text{FE}_3$ & $\text{FD}_1$ & $\text{FD}_2$ & $\text{FD}_3$ \\
\hline
$\beta_L$ & $\underset{(0.0231)}{0.5681^{***}}$ & $\underset{(0.0162)}{0.6479^{***}}$ & $\underset{(0.0431)}{0.5408^{***}}$ & $\underset{(0.0294)}{0.5484^{***}}$ & $\underset{(0.0294)}{0.5473^{***}}$ & $\underset{(0.0293)}{0.5483^{***}}$ \\
$\beta_K$ & $\underset{(0.0134)}{0.1495^{***}}$ & $\underset{(0.0231)}{0.0210}$ & $\underset{(0.0375)}{0.0280}$ & $\underset{(0.0232)}{0.0629^{**}}$ & $\underset{(0.0234)}{0.0612^{**}}$ & $\underset{(0.0241)}{0.0565^{**}}$ \\
$\delta_L$ & $\underset{(0.0225)}{0.1532^{***}}$ & -- & $\underset{(0.0283)}{0.1419^{***}}$ & $\underset{(0.0011)}{-0.0002}$ & -- & $\underset{(0.0030)}{0.0045}$ \\
$\delta_K$ & -- & $\underset{(0.0258)}{0.1793^{***}}$ & $\underset{(0.0457)}{0.1667^{***}}$ & -- & $\underset{(0.0009)}{-0.0009}$ & $\underset{(0.0026)}{-0.0046^{*}}$ \\
\hline
$H_0: \delta_L = \delta_K = 0$ &  &  & $\underset{p=0.000}{44.111}$ &  &  & $\underset{p=0.182}{3.406}$ \\
\hline
\end{tabular}

\begin{flushleft}\footnotesize
Notes: Robust standard errors in parentheses. ${}^{***}p<0.01$, ${}^{**}p<0.05$, ${}^{*}p<0.10$.  
“Joint test” reports the Wald statistic for the joint significance of $\delta_L$ and $\delta_K$ in columns $\text{FE}_3$ and $\text{FD}_3$; the $p$-value is shown below each statistic using $\underset{\cdot}{\cdot}$. The Wald test is asymptotically $\chi^2(2)$ under the null of two linear restrictions $\delta_L=\delta_K=0$.
\end{flushleft}
\end{table}


While the above suggest that both estimators are inconsistent and not asymptotically normal, we continue with the main hypothesis test of the assignment as if they were. A proper paper would have to employ different models that can account for the endogeneity problem. 

We proceed to test the null hypothesis of constant returns to scale. From Table \ref{tab:wald}, we reject the null at the 1 \% significance level with a p-value of $\approx 0$. Thus, we conclude that the (Cobb-Douglas) production function does not exhibit constant returns to scale, for French manufacturing firms. 

\begin{table}[H]
\centering
\caption{Robust Wald Test for CRS in CD}
\label{tab:wald}
\begin{tabular}{cccc}
\toprule
Wald stat & df & $\chi^2_{1^{(0.95)}}$ & p-value \\
\midrule
19.403 & 1 & 3.841 & 0.000 \\
\bottomrule
\end{tabular}
\end{table}

Naturally one might then be interested in testing whether the production function exhibits increasing or decreasing returns to scale. This can be done by constructing one-sided tests where we are looking for a rejection of the null, due to the boundary case of CRS being included in the null in which results are inconclusive. To test for decreasing return to scale we test: $H_0:\beta_K + \beta_L \geq 1$ (increasing or constant RTS) against the alternative of $H_A:\beta_K + \beta_L < 1$ (decreasing RTS) or vice versa for increasing RTS. Since the squared $t$-statistic yields the same numerical result as the Wald stat in table \ref{tab:wald}, the only difference for increasing RTS and decreasing RTS is the \textit{critical value} region, for the $t$-statistic, have a different sign (+) for increasing and (-) for decreasing. The results are $t=\pm \sqrt{(19.403)}=\pm4.404$. Inserting the estimates from our FE model yields $t=-4.404$, which illustrates the degree of missing information in not knowing the sign going from Wald to t-test. 

%Our degrees of freedom are $N=441-1=440$ but because of asymptotic properties when $N\rightarrow\infty$ the $t(N)\xrightarrow{d} \mathcal{N}(0,1)$. Therefore the critical values for our $t$-distribution are almost identical to the critical values for a normal distribution, for a two-sided and one-sided test respectively. 

For a one-sided test with decreasing RTS, the critical value region is $(-\infty,-1.648]$. Since the \textit{t}-statistic is within the critical region, we reject the $H_0$ of constant or increasing returns to scale. As such we cannot reject the $H_A$ of decreasing returns to scale. The $H_0$ for constant or decreasing return to scale have the opposite critical region $[1.648,\infty)$ where the $t$-statistic does not lie within. We reach the same conclusion for the FD model, with $t\approx-12.249$. As such the French manufacturing firms exhibit decreasing returns to scale, under the assumption that \textbf{FE(D).1} and \textbf{FE.2/FD.2} hold.

%Importantly we need to consider that the test includes the boundary i.e. $\beta_K + \beta_L = 1$ under the null such that we cant draw a conclusion if the null is not rejected, which is why the tests are read somewhat opposite of what one might expect. One might suspect that this particular sector exhibits increasing returns to scale, as firms in manufacturing often benefit from economies of scale, which indeed is the case as seen from Table \ref{tab:rts_tests}.
% \begin{table}[H]
\centering
\caption{One-Sided Tests for Returns to Scale}
\label{tab:rts_tests}
\begin{tabular}{lcccc}
\toprule
Hypothesis & Estimate & t-stat & p-value & Reject $H_0$ \\
\midrule
Increasing RTS ($\beta_K + \beta_L > 1$) & 0.849 & -11.627 & 1.000 & No \\
Decreasing RTS ($\beta_K + \beta_L < 1$) & 0.849 & -11.627 & 0.000 & Yes \\
\bottomrule
\end{tabular}
\end{table}

