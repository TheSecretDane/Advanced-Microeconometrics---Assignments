\section{Empirical results}

The data consists of $N = 441$ French manufacturing firms over 12 years $(T=12)$ from 1968 to 1979, and is thus a panel data set with $NT = 5292$ observations. The data is balanced. 

Maybe variable def. 

We perform the two-way or double-demean transformation of our data matrices as described in the methodology section (Ref). The estimates along with standard errors and $p$-values are reported in Table \ref{tab:results}. 

Next we proceed to test the null hypothesis of constant returns to scale. We reject the null at the 1 \% significance level with a p-value of $\approx 0$. Thus, we conclude that the (Cobb-Douglas) production function does not exhibit constant returns to scale, for French manufacturing firms in the period 1968-1979. 

In addition one might want to test whether the production function exhibits increasing or decreasing returns to scale. This can be done by testing the null hypothesis of $\beta_K + \beta_L \geq 1$ against the alternative of $\beta_K + \beta_L < 1$ (decreasing returns to scale) or vice versa for increasing returns to scale. One might suspect that this particular sector exhibits increasing returns to scale, as firms in manufacturing often benefit from economies of scale. 