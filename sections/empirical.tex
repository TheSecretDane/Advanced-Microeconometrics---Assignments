\section{Empirical results}

The data consists of $N = 441$ French manufacturing firms over 12 years $(T=12)$ from 1968 to 1979, and is thus a panel data set with $NT = 5292$ observations. The data is balanced.  

We perform the within transformation of our data matrices as described in the methodology section (Ref) and perform pooled OLS on (ref est eq). The estimates along with standard errors and t-values are reported in Table \ref{tab:estimates}. 

\begin{table}[H] 
\centering
\caption{Fixed Effects Regression Results}
\begin{tabular}{lcccc}
\toprule
 & $\beta$  & Se & t-values & p-value \\
\midrule
$\ell $  & 0.6942 & 0.0147 & 47.2447 & 0.0000 \\
$k$ & 0.1546 & 0.0130 & 11.9311 & 0.0000 \\
\midrule
$R^2$ & \multicolumn{4}{l}{0.477} \\
$\sigma^2$ & \multicolumn{4}{l}{0.018} \\
\bottomrule
\end{tabular}
\label{tab:results}
\end{table}


Next we proceed to test the null hypothesis of constant returns to scale. From Table \ref{tab:wald}, we reject the null at the 1 \% significance level with a p-value of $\approx 0$. Thus, we conclude that the (Cobb-Douglas) production function does not exhibit constant returns to scale, for French manufacturing firms in the period 1968-1979. 

\begin{table}[H]
\centering
\caption{Wald Test Results}
\label{tab:wald}
\begin{tabular}{cccc}
\toprule
Wald stat & df & $\chi^2_{1^{(0.95)}}$ & p-value \\
\midrule
135.190 & 1 & 3.842 & 0.000 \\
\bottomrule
\end{tabular}
\end{table}


In addition one might want to test whether the production function exhibits increasing or decreasing returns to scale. This can be done by testing the null hypothesis of $\beta_K + \beta_L \geq 1$ against the alternative of $\beta_K + \beta_L < 1$ (decreasing returns to scale) or vice versa for increasing returns to scale. Importantly we need to consider that the test includes the boundary i.e. $\beta_K + \beta_L = 1$ such that we cant draw a conclusion if the null is accepted, which is why the tests are read somewhat opposite of what one might expect. One might suspect that this particular sector exhibits increasing returns to scale, as firms in manufacturing often benefit from economies of scale, which indeed is the case as seen from Table \ref{tab:rts_tests}.

\begin{table}[H]
\centering
\caption{One-Sided Tests for Returns to Scale}
\label{tab:rts_tests}
\begin{tabular}{lcccc}
\toprule
Hypothesis & Estimate & t-stat & p-value & Reject $H_0$ \\
\midrule
Increasing RTS ($\beta_K + \beta_L > 1$) & 0.849 & -11.627 & 1.000 & No \\
Decreasing RTS ($\beta_K + \beta_L < 1$) & 0.849 & -11.627 & 0.000 & Yes \\
\bottomrule
\end{tabular}
\end{table}

