\section{Methodology}

\subsection{(Econometric theory and methodology)}

Define $\pmb{x}_{it}=[k_{it}, l_{it}]$ as the regressors of our model and $y_{it}$ as the dependent variable, which are both observable. Further define the composite error term $v_{it}=c_i+u_{it}$, which contains both a time-invariant term $c_i$ and a time-varying term $u_{it}$. Further define 

\subsubsection*{Pooled OLS}
A naive approach to formulate an econometric model would be to use pooled OLS for the case when we define the composite error term $u_{it}=v_{it}+\epsilon_{it}$
\begin{align*}
    y_{it} &= \pmb{x}_{it} \pmb{\beta}  +u_{it} \\
    \Leftrightarrow \pmb{x}_{it}'y_{it}&=\pmb{x}_{it}'\pmb{x}_{it}\pmb{\beta} +\pmb{x}_{it}'u_{it} \\
    \Leftrightarrow \mathbb{E}[\pmb{x}_{it}'y_{it}] &=\mathbb{E}[\pmb{x}_{it}'\pmb{x}_{it}] \pmb{\beta}  + \mathbb{E}[\pmb{x}_{it}'u_{it}] \\ \Leftrightarrow  \pmb{\beta} &= (\mathbb{E}[\pmb{x}_{it}'\pmb{x}_{it}])^{-1} \left(  \mathbb{E}[\pmb{x}_{it}'y_{it}]-\mathbb{E}[\pmb{x}_{it}' u_{it}]\right)
\end{align*}
It is evident that identification of $\pmb{\beta}$ requires that the composite error $u_{it}$ is uncorrelated with the regressor. Unpacking the composite error such that $\mathbb{E}[\pmb{x}_{it}'u_{it}]=\mathbb{E}[\pmb{x}_{it}'v_{it}]+\mathbb{E}[\pmb{x}_{it}'\epsilon_{it}]$ shows that logged TFP (time-invariant unobservable) $v_{it}$ must be assumed uncorrelated with logged capital and logged employment (regressors) in addition to $\mathbb{E}[\pmb{x}_{it} \epsilon_{it}]=0$, to successfully identify $\pmb{\beta}$. An example of violation of $\mathbb{E}[\pmb{x}_{it}'v_{it}]=0$ would be if firms that have more capital (or employees) are generally more productive than firms with less capital (or less employees). This could be due to structural differences a.o. 

If we were willing to assume $\mathbb{E}[\pmb{x}_{it}'v_{it}]=0$, then to estimate $\pmb{\beta}$ using pooled OLS, we would stack $y_{it}$ and $\pmb{x}_{it}$ over $(i,t)$ into matrices, obtaining $\pmb{X}$ and $\pmb{y}$, both two-dimensional of size $(NT,K)$ and $(NT,1)$ respectively.
$$\hat{\pmb{\beta}} = (\pmb{X}' \pmb{X})^{-1}\pmb{X}'\pmb{y}$$
However with a DGP as $\pmb{y}=\pmb{X}\pmb{\beta}+\pmb{v}+\pmb{\epsilon}$, we would obtain a biased estimate since:
\begin{align*}
    \hat{\pmb{\beta}} &= (\pmb{X}'\pmb{X})^{-1} (\pmb{X}'\pmb{X}\pmb{\beta} +\pmb{X}'\pmb{v}+\pmb{X}'\pmb{\epsilon})=\pmb{\beta}+\pmb{X}'\pmb{v} +\
\end{align*}
(Ser lige hvad jeg gør med det her afsnit, skal nok skærre fra, strømline)

\subsubsection*{Fixed Effects}
We now proceed to consider another estimator, namely the fixed effects estimator. Firstly, the data is transformed by a within-transformation, which demeans the data. In that regard, we denote the time-demeaned data with double dots (mere elegant), which results in the following equation for our model $\forall \ t,i$.
\begin{align*}
    y_{it}-\bar{y}_{it} &= \beta(\pmb{x}_{it}-\bar{\pmb{x}}_{it})+(c_i-c_i)+(u_{it} - \bar{u}_{it}) \\
    \ddot{y}_{it} &= \beta \ddot{\pmb{x}}_{it}+\ddot{u}_{it}
\end{align*}
We then consider the identification of $\beta$. Firstly, the data is stacked over time time-dimension (why is this smart?= så man slipper for at skrive summer) such that $\ddot{\pmb{y}}_{i}$ and $\ddot{\pmb{u}}_{i}$ are $T \times 1$ vectors and $\pmb{\ddot{X}}$ is a $T\times K$ matrix.
\begin{align*}
    \ddot{\pmb{y}}_{i} &= \pmb{\ddot{X}}_{i} \pmb{\beta}  +\pmb{\ddot{u}}_{i} \\
    \Leftrightarrow \pmb{\ddot{X}}_{i}'\ddot{\pmb{y}}_{i}&=\pmb{\ddot{X}}_{i}'\pmb{\ddot{X}}_{i}\pmb{\beta} +\pmb{\ddot{X}}_{i}'\pmb{\ddot{u}}_{i} \\
    \Leftrightarrow \mathbb{E}[\pmb{\ddot{X}}_{i}'\pmb{\ddot{y}}_{i}] &=\mathbb{E}[\pmb{\ddot{X}}_{i}'\pmb{\ddot{X}}_{i}] \pmb{\beta}  + \mathbb{E}[\pmb{\ddot{X}}_{i}'\pmb{\ddot{u}}_{i}] \\ \Leftrightarrow  \pmb{\beta} &= (\mathbb{E}[\pmb{\ddot{X}}_{i}'\pmb{\ddot{X}}_{i}])^{-1} \left(  \mathbb{E}[\pmb{\ddot{X}}_{i}'\pmb{\ddot{y}}_{i}]-\mathbb{E}[\pmb{\ddot{X}}_{i}' \pmb{\ddot{u}}_{i}]\right)
\end{align*}
, i.e. we see that $\pmb{\beta}$ is exactly identified if we assume the following:
\begin{enumerate}
    \item \underline{Strict exogenity} $\mathbb{E}[\pmb{\ddot{u}}_{i}\vert \pmb{x}_{i1},\pmb{x}_{i2},\dots,\pmb{x}_{iT},c_i]=0$, which implies that the time-demeaned regressors and error terms are uncorrelated $\mathbb{E}[\pmb{\ddot{X}}_{i}' \pmb{\ddot{u}}_{i}]=0$ (derive + interpret?). 
    \item the matrix $\mathbb{E}[\pmb{\ddot{X}}_{i}'\pmb{\ddot{X}}_{i}]$ fulfills a \underline{ (full) rank condition} such that is has full rank, i.e. $rank(\mathbb{E}[\pmb{\ddot{X}}_{i}'\pmb{\ddot{X}}_{i}])=K$ and can be inverted. Interpret
\end{enumerate}
which would imply:
$$\mathbb{E}[\pmb{\ddot{X}}_{i}' \pmb{\ddot{u}}_{i}]= 0\Rightarrow \pmb{\beta} = (\mathbb{E}[\pmb{\ddot{X}}_{i}'\pmb{\ddot{X}}_{i}])^{-1} \mathbb{E}[\pmb{\ddot{X}}_{i}'\pmb{y}_{i}]$$
Using the analogy principle, estimation of $\pmb{\beta}$ would result in the following estimator $\pmb{\hat{\beta}}_{FE}$, where we use that the expectation is over the individuals $i$.
$$\pmb{\hat{\beta}}_{FE}= \left( \frac{1}{N} \sum_{i=1}^N {\pmb{\ddot{X'}}_{i} \pmb{\ddot{X}}_{i} } \right) ^{-1}\frac{1}{N} \sum_{i=1}^N {\pmb{\ddot{X'}}_{i} \pmb{\ddot{y}}_{i} }$$

With $\pmb{\ddot{X}}'\pmb{\ddot{X}}$, we obtain a $(K \times K)$ matrix, which, with a $(K \times 1)$ vector from $\pmb{\ddot{X'}}_{i} \pmb{\ddot{y}}_{i}$ results in a $(K \times 1)$ vector of the parameters.

\subsubsection*{Consistency of Fixed Effects estimator}
Now, we turn to the consistency of the FE-estimator. Inserting $\pmb{\ddot{y}}_i$ in the expression for $\pmb{\hat{\beta}}_{FE}$, we obtain the following:
\begin{align*}
    \Rightarrow \pmb{\hat{\beta}}_{FE} &= \left( \frac{1}{N} \sum_{i=1}^N {\pmb{\ddot{X'}}_{i} \pmb{\ddot{X}}_{i} } \right) ^{-1} \left(\frac{1}{N} \sum_{i=1}^N {\pmb{\ddot{X'}}_{i}}\pmb{\ddot{X}}_{i} \pmb{\beta}+\pmb{\ddot{X'}}_{i}\pmb{\ddot{u}}_i \right) \\
    &= \pmb{\beta} +\left( \frac{1}{N} \sum_{i=1}^N {\pmb{\ddot{X'}}_{i} \pmb{\ddot{X}}_{i} } \right) ^{-1} \left( \frac{1}{N} \sum_{i=1}^N {\pmb{\ddot{X}}_i' \pmb{\ddot{u}}_i} \right)
\end{align*}
The Law of Large Numbers (LLN) implies that $p\text{-lim}\left(\frac{1}{N} \sum_{i=1}^N {\pmb{\ddot{X'}}_{i} \pmb{\ddot{X}}_{i} } \right)=\mathbb{E}[\pmb{\ddot{X'}}_{i} \pmb{\ddot{X}}_{i}]$ and $p\text{-lim}\left(\frac{1}{N} \sum_{i=1}^N {\pmb{\ddot{X'}}_{i} \pmb{\ddot{u}}_{i} } \right)=\mathbb{E}[\pmb{\ddot{X'}}_{i} \pmb{\ddot{u}}_{i}]$. Then we use Slutskys theorem: $p\text{-lim}\left( \left(\frac{1}{N} \sum_{i=1}^N {\pmb{\ddot{X'}}_{i} \pmb{\ddot{X}}_{i} } \right)^{-1}\right)=(\mathbb{E}[\pmb{\ddot{X'}}_{i} \pmb{\ddot{X}}_{i}])^{-1}$, which is allowed since the inverse matrix is defined, due to the rank assumption, and the inverse operator is a continuous function. Altogether, it implies that the FE-estimator $\pmb{\hat{\beta}}_{FE}$ is consistent, which is a first-order concern.

\subsubsection{Asymptotic normality of Fixed Effects estimator}
When analysing our consistent FE-estimator, we need to know the asymptotic distribution of the parameter. Rearranging the equation from the former section, we get the following:
\begin{align*}
    \sqrt{N}(\pmb{\hat{\beta}}_{FE}-\pmb{\beta})&=  \left( \frac{1}{N} \sum_{i=1}^N {\pmb{\ddot{X'}}_{i} \pmb{\ddot{X}}_{i} } \right) ^{-1} \left( \frac{1}{\sqrt{N}} \sum_{i=1}^N {\pmb{\ddot{X}}_i' \pmb{\ddot{u}}_i} \right)
\end{align*}
From the former section, we saw that the Law of Large Numbers and Slutskys Theorem combined with our assumptions (strict exogeneity and full rank condition) implied that asymptotically $\mathbb{E}[\sqrt{N}(\pmb{\hat{\beta}}_{FE}-\pmb{\beta})]=\pmb{0}$. This means that the sum of real-valued i.i.d. r.v.s converges accordingly to the Central Limit Theorem such that $\left( \frac{1}{\sqrt{N}} \sum_{i=1}^N {\pmb{\ddot{X}}_i' \pmb{\ddot{u}}_i} \right)\rightarrow _d \text{N}(\pmb{0},\mathbb{E}[\pmb{\ddot{X}}_i' \pmb{\ddot{u}}_i(\pmb{\ddot{X}}_i' \pmb{\ddot{u}}_i)'])=^d \text{N}(\pmb{0},\mathbb{E}[\pmb{\ddot{X}}_i' \pmb{\ddot{u}}_i\pmb{\ddot{u}}_i'\pmb{\ddot{X}}_i])$.

Since we have convergence in probability $p\text{-lim}\left( \left(\frac{1}{N} \sum_{i=1}^N {\pmb{\ddot{X'}}_{i} \pmb{\ddot{X}}_{i} } \right)^{-1}\right)=(\mathbb{E}[\pmb{\ddot{X'}}_{i} \pmb{\ddot{X}}_{i}])^{-1}$ for the first term and convergence in distribution for the second term, the product rule implies that the expression for $\sqrt{N}(\pmb{\hat{\beta}}_{FE}-\pmb{\beta})$ converges only in the distributional sense such that:
$$\Rightarrow \sqrt{N}(\pmb{\hat{\beta}}_{FE}-\pmb{\beta}) \sim \text{N} \left(\pmb{0},
(\mathbb{E}[\pmb{\ddot{X'}}_{i} \pmb{\ddot{X}}_{i}])^{-1} \mathbb{E}[\pmb{\ddot{X}}_i' \pmb{\ddot{u}}_i\pmb{\ddot{u}}_i'\pmb{\ddot{X}}_i] (\mathbb{E}[\pmb{\ddot{X'}}_{i} \pmb{\ddot{X}}_{i}])^{-1} \right)$$

\subsection{Efficiency}

It follows, that for the FE estimator to be efficient - i.e. have the smallest possible (asymptotic) variance -, we need to make some assumptions regarding the variance of the idiosyncratic error term, $\varepsilon_{it}$. Specifically, we need that the variance is homoscedastic and not serially correlated, i.e. $\text{Var}(\varepsilon_{it}) = \sigma^2$ and $\text{Cov}(\varepsilon_{it}, \varepsilon_{is}) = 0$ for $t \neq s$, or in the stacked notation, that, $\Omega_{\bm{\varepsilon}} = \sigma^2 I_T$. This results in the following (ref Mikkels eq for asymptotic normality), 
\begin{align*}
    \mathbb{E}\left[ \bm{\ddot{X}}_i^\prime \bm{u}_i \bm{u}_i^\prime \bm{\ddot{X}}_i \right] &= \mathbb{E}\left[ \bm{\ddot{X}}_i^\prime \mathbb{E}[\bm{u}_i \bm{u}_i^\prime | \bm{x}_i \alpha_i]\bm{\ddot{X}}_i \right] \\
    &= \mathbb{E}[\ddot{X}_i^\prime \sigma_u^2 \bm{I}_T \bm{\ddot{X}}_i]
\end{align*}
such that the asymptotic variance is,
\begin{align*}
    \text{Avar}(\hat{\beta}_{FE}) = \sigma_u^2 \mathbb{E}[\bm{\ddot{X}}_i^\prime \bm{\ddot{X}}_i]^{-1} / N
\end{align*}
Using the analogy principle, we have, 
\begin{align*}
    \hat{\text{Avar}}(\hat{\beta}_{FE}) = \hat{\sigma}_u^2 \left( \sum_{i=1}^N \bm{\ddot{X}}_i^\prime \bm{\ddot{X}}_i \right)^{-1}, \quad &\text{where} \quad \hat{\sigma}_u^2 = \frac{1}{N(T-1) - K} \sum_{i=1}^N \hat{\bm{u}}_i^\prime \hat{\bm{u}}_i, \\
    &\text{and} \quad \hat{\bm{u}}_i = \bm{\ddot{y}}_i - \bm{\ddot{X}}_i \hat{\beta}_{FE}
\end{align*}
Note that we need to correct the degrees of freedom for the implicit estimation of the fixed effects, which is why we subtract $N$ from the denominator. If these assumptions do not hold, we can still obtain consistent estimates of the standard errors using robust or clustered standard errors, though often at the cost of efficiency. In which case we estimate the middle part of the sandwich from (Mikkel ref normality) using the analogy principle i.e. using the estimated residuals, such that, 
\begin{align*}
    \hat{\text{Avar}}(\hat{\beta}_{FE}) = \left( \sum_{i=1}^N \bm{\ddot{X}}_i^\prime \bm{\ddot{X}}_i \right)^{-1} \left( \sum_{i=1}^N \bm{\ddot{X}}_i^\prime \hat{\bm{u}}_i \hat{\bm{u}}_i^\prime \bm{\ddot{X}}_i \right) \left( \sum_{i=1}^N \bm{\ddot{X}}_i^\prime \bm{\ddot{X}}_i \right)^{-1}
\end{align*} 
Which is consistent under just the strict exogeneity assumption as well as the rank assumption - because consistent estimation of the error, as well as the outer matrices being invertible? (remove outer sums possibly and go full stacked like a maniac). 

\subsubsection*{Wald-test}
%(We could also test for joint significance for the year dummies) \newline
%(Rækkefølgen skal ændres på, da nogle antagelser bruges i de tidligere underafsnit) \newline
We utilise the Wald-test to test for CRS, where the null-hypothesis is $H_0: \mathbf{R}\boldsymbol{\beta}=\mathbf{r}$. Under weak regularity conditions - consistency of the FE estimator, asymptotic normality and $\mathbf{R}$ has full rank (\textit{refer to  consistency, and asymptotic normality equations}), the Wald statistic can be written as:


\begin{equation*}
    W:=(\mathbf{R}\widehat{\boldsymbol{\beta}}-\mathbf{r})[\mathbf{R}\widehat{\text{Avar}(\widehat{\beta}_{FE})}\mathbf{R}']^{-1}(\mathbf{R} \boldsymbol{\widehat{\beta}}-\mathbf{r})
\end{equation*}

Under the $H_0$ the Wald statistic follows a $\chi^2_Q$ distribution. The 
When testing for CRS we test $H_0:\beta_K+\beta_K=1$, where we have $Q=1$ degrees of freedom and  $\mathbf{R}$ is a $[1 \times K]$ matrix and $\textbf{r}=1 \times 1$ matrix, where $K$ is the rank $K=E[\textbf{x}'\textbf{x}]$  from assumption FE.2. If we use normal standard errors, we require stronger assumptions where under assumption FE.3 the FE estimator is asymptotically efficient $\rightarrow$ we can use normal standard errors. If we relax assumption FE.3, or simply reject the null-hypothesis from a serial-correlation test, like a placebo test with an AR(1) model, (\textit{see section for serial correlation for this specific test}) we have to use robust standard errors. The only change in the Wald-statistic is the variance-covariance matrix $\text{Avar}(\hat{\beta}_{FE})$, where the robust variance-covariance matrix is given by the sandwich formula. \newline
\textit{As seen under the efficiency section. We need to differentiate between the normal and robust version of Avar($\beta$)}.

%(\textit{Currently we have used normal standard errors, but maybe we should consider clustered standard errors to account for within-firm serial correlation. I'll rewrite this, if we have implemented clustered into our results, but the implementation is not done in our exercises. Only normal and robust errors are implemented.}

\subsubsection*{Serial correlation test}
A test for serial correlation can imply several tests for any type of serial correlation. In this report we utilise an autoregressive process with one lag (AR(1)) to test whether there is a significance from a leading explanatory variable in t+1 on the auxiliary regression. This can be defined shown by:

\subsubsection*{Homoskedasticity (maybe)}

